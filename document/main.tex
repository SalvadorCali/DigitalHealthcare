% !TEX root = main.tex
% !TEX spellcheck = it-IT
\documentclass[12pt,a4paper,twoside,openright,titlepage]{book}	
	
\usepackage[binding=5mm]{layaureo}
\usepackage[swapnames]{frontespizio}
\usepackage[english,italian]{babel}
\usepackage{microtype}
\usepackage{hyperref}
\usepackage{tabularx}
\usepackage{listings}
\usepackage{float}
\usepackage{changepage,calc}
\usepackage{emptypage}
\usepackage{indentfirst}
\usepackage{fancyhdr}
\usepackage{relsize}
\usepackage{bookmark}
\usepackage{lipsum}	
\usepackage[footnote,smaller]{acronym}
\usepackage{pdfpages}
\setlength{\headheight}{15pt}
\usepackage{graphicx}

\graphicspath{ {images/} }

\usepackage[autostyle,italian=guillemets]{csquotes}
\usepackage[style=philosophy-modern,hyperref,backref,natbib,backend=biber,defernumbers=true]{biblatex}
\addbibresource{Bibliografia.bib}

%\title{Library Box}
%\date{2020-12-10}
%\author{Andrea Calici (10490117)}

\begin{document}
%% !TEX root = Tesi.tex
% !TEX spellcheck = it-IT
\newcommand{\myName}{Andrea Calici}
\newcommand{\myMatricola}{944717}
\newcommand{\myTitle}{Titolo della Tesi}
\newcommand{\myUni}{Politecnico di Milano}
\newcommand{\myFaculty}{Scuola di Ingegneria Industriale e dell'Informazione}
\newcommand{\myDegree}{Computer Science and Engineering}
\newcommand{\myThesis}{Tesi di Laurea Magistrale}
\newcommand{\myDepartment}{Dipartimento di Informatica}
\newcommand{\myProf}{Prof.~Luciano~Baresi}
\newcommand{\myLocation}{Milano}
\newcommand{\myTime}{Aprile 2014}
\newcommand{\myAcademicYear}{2020--2021}
\newcommand{\myUrlUni}{www.polimi.it}
\newcommand{\myUrlFaculty}{www.ingindinf.polimi.it}

\newenvironment{sistema}%
	{\left\lbrace\begin{array}{@{}l@{}}}%
	{\end{array}\right.}
%
% epsilon theta rho phi
\renewcommand{\epsilon}{\varepsilon}
\renewcommand{\theta}{\vartheta}
%\renewcommand{\rho}{\varrho}
\renewcommand{\phi}{\varphi}
%
\renewcommand{\vec}{\mathbf} 	% vettori in tondo nero
%

% Impostazioni degli acronimi
\makeatletter
\def\bflabel#1{{\textbf{\textsf{#1}}\hfill}}
\renewenvironment{AC@deflist}[1]%
{\ifAC@nolist%
\else%
\begin{list}{}%
{\settowidth{\labelwidth}{\textbf{\textsf{#1}}}%
\setlength{\leftmargin}{\labelwidth}%
\addtolength{\leftmargin}{\labelsep}%
\renewcommand{\makelabel}{\bflabel}}%
\fi}%
{\ifAC@nolist%
\else%
\end{list}%
\fi}%
\makeatother

\hyphenation{OpenFOAM}
\hyphenation{Matlab}
\hyphenation{bash}

\includepdf{fronte/fronte.pdf}
%\pdfbookmark[1]{Frontespizio}{Frontespizio}
\begin{frontespizio}

\Preambolo{\renewcommand{\frontsmallfont}[1]{\small Matr.}}
\Margini{1.5cm}{1.5cm}{1.5cm}{1.5cm}
\Istituzione{Politecnico di Milano}
\Logo[2.5cm]{images/logo}
\Divisione{Scuola di Ingegneria Industriale e dell'Informazione}
%\Dipartimento{Meccanica}
\Corso{Computer Science and Engineering}
\Titoletto{Tesi di Laurea Magistrale}
\Titolo{Progetto Salvavita}
\Candidato[944717]{Andrea Calici}
\Relatore{Prof.~Luciano~Baresi}
\Annoaccademico{2020--2021}
\Punteggiatura{}
\Rientro{1cm}
\end{frontespizio}
\pagenumbering{gobble}
%\maketitle
% !TEX root = ../Tesi.tex
% !TEX spellcheck = it-IT
%\tableofcontents
\cleardoublepage
%
% ------------------------------------------------------------------------ %
%
% Indice Generale
%
\pdfbookmark{\contentsname}{tableofcontents}
%
\setcounter{tocdepth}{2}
%
\tableofcontents
%
\cleardoublepage
%
% ------------------------------------------------------------------------ %
%
% Indice delle Figure
%
\phantomsection
%
\pdfbookmark{\listfigurename}{lof}
%
\listoffigures
%
\cleardoublepage
%
% ------------------------------------------------------------------------ %
%
% Indice delle Tabelle
%
\phantomsection
%
\pdfbookmark{\listtablename}{lot}
%
\listoftables
%
\cleardoublepage
%
% ------------------------------------------------------------------------ %
%
% Indice dei Listati di Programma
%
\phantomsection
%
\pdfbookmark{\lstlistlistingname}{lol}
%
\lstlistoflistings
%
\cleardoublepage
% !TEX root = ../Tesi.tex
% !TEX spellcheck = it-IT
\cleardoublepage
\phantomsection
\pdfbookmark{Ringraziamenti}{ringraziamenti}
\chapter*{Ringraziamenti}
Prova

\medskip

Desidero inoltre ringraziare esplicitamente:
\begin{description}
\item[{\scshape Esplicito1}] per vari motivi;
\item[{\scshape Esplicito2}] per altri motivi;
\item[{\scshape Esplicito3}] per puro piacere, senza particolari motivi.
\end{description}

\bigskip
 
\noindent\textit{Prova}
\hfill A.~C.
% !TEX root = ../Tesi.tex
% !TEX spellcheck = it-IT
% 
\cleardoublepage
%
\phantomsection
%
\pdfbookmark{Sommario}{Sommario}

\begingroup
%\let\clearpage\relax
\let\cleardoublepage\relax
\let\cleardoublepage\relax
% ------------------------------------------------------------------------ %
%
\chapter*{Sommario}
%
Prova

\medskip
%
\noindent \textbf{Parole chiave:} 
PoliMi,
Tesi,
LaTeX,
Scribd
%
\clearpage
%\vfill
%
% ------------------------------------------------------------------------ %
%
\selectlanguage{english}
%
\pdfbookmark{Abstract}{Abstract}
%
\chapter*{Abstract}
%
Text of the abstract in english\dots\\
Prova

\medskip
%
\noindent \textbf{Keywords:} 
PoliMi,
Master Thesis,
LaTeX,
Scribd

\selectlanguage{italian}

\endgroup			
%
%\vfill
%\input{fronte/acronimi.tex}

\addcontentsline{toc}{chapter}{Definizioni e Acronimi}
\chapter*{Definizioni e Acronimi}
\section*{Definizioni}
In questa sezione verrà presentata una lista di termini comuni e di utile comprensione che verranno utilizzati nel corso della trattazione:
\begin{itemize}
\item Utente: il generico utente finale dell'applicazione e comprensivo delle figure del cittadino e del volontario;
\item Cittadino: utente finale che è in grado di accedere ai propri dati sanitari e salvavita e generare attraverso l'applicazione documenti di utilità;
\item Volontario: utente finale dell'applicazione che rappresenta la figura di riferimento per l'inserimento dei dati di natura pratica del cittadino e che può avere accesso al sistema sia per la fase di inserimento di informazioni, sia per eseguire funzioni di utilizzo dei dati;
\item Carta d'Identità Salvavita: documento contenente il codice QR salvavita e i dati sanitari di emergenza del cittadino;
\item Fascicolo Sanitario Elettronico: strumento che permette di ricostruire la storia clinica del cittadino con la presenza dei dati e dei documenti sanitari dello stesso;
\item Scheda Sanitaria Individuale: documento compilato dal Medico di Medicina Generale che contiene la storia clinica del paziente e le sue informazioni sanitarie.
\end{itemize}

\section*{Acronimi}
In questa sezione verrà presentata una lista di acronimi comuni e di utile comprensione che verranno utilizzati nel corso della trattazione:
\begin{itemize}
\item CIS: Carta d'Identità Salvavita;
\item MMG: Medico di Medicina Generale;
\item ICE: In Caso di Emergenza;
\item MVI: Medici Volontari Italiani;
\item FSE: Fascicolo Sanitario Elettronico;
\item SSI: Scheda Sanitaria Individuale;
\end{itemize}
\cleardoublepage

\pagenumbering{arabic}
%\addcontentsline{toc}{chapter}{Introduzione}
\chapter{Introduzione}
\markboth{Introduzione}{Introduzione}	% headings
%
\label{cap:introduzione}
\section{Panoramica Generale}
Lo scopo di questo documento è di descrivere il processo di design e di implementazione dell'applicazione mobile e web sviluppata per Android e per i più comuni web browsers attualmente disponibili, attraverso la descrizione funzionale dell'intero sistema, l'analisi dettagliata dell'architettura, i componenti con le relative interazioni e i più frequenti casi d'uso.\newline
Partendo dal lavoro svolto da Medici Volontari Italiani che ha sviluppato la Carta d'Identità Salvavita e il progetto Busta Rossa del Comune di Milano volto alla salvaguardia della salute del cittadino, io e Davide Laffi abbiamo sviluppato un sistema in grado di salvare le informazioni sanitarie dei cittadini inserite dal Medico di Medicina Generale e dal volontario incaricato in modo da poter essere accessibili al relativo cittadino tramite l'uso di un'applicazione mobile.\newline
In particolare, per soddisfare questi requisiti, il sistema complessivo si compone di 3 parti fondamentali che permettono di gestire in maniera indipendente la fase di inserimento dati da quella di utilizzo, riuscendo al tempo stesso ad interfacciarsi perfettamente. Il lato server dell'applicazione, sviluppato su Google Firebase, offre funzionalità di database che servono allo scopo appena descritto e permettono di integrare le due altre componenti del sistema: l'applicazione web e l'applicazione mobile. Per quanto riguarda il lato web, il suo utilizzo è riservato agli MMG e ai volontari che sono in grado di inserire attraverso un form completo i dati sanitari e salvavita relativi al cittadino e modificarli all'occorrenza. Queste informazioni, salvate nel database, sono quindi accessibili dall'utente finale, il cittadino possessore di tali dati, attraverso l'applicazione mobile che, oltre ad offire una panoramica sugli stessi, permette anche la creazione di documenti di utilità come ad esempio il Profilo Sanitario Sintetico o la Carta d'Identità Salvavita. In questo modo il cittadino, il cui unico requisito è avere un account per eseguire l'accesso all'applicazione, avrà accesso in ogni momento ai suoi dati sanitari e salvavita che saranno utilizzabili sia in situazioni di necessità, sia in situazioni di emergenza nel caso in cui il soccorritore avesse bisogno di tali informazioni per intervenire tempestivamente nel soccorso del cittadino stesso.

\section{Stato dell'Arte}
Lo stato attuale dell'arte presenta per il singolo cittadino una serie di entità differenziate in grado di gestire i suoi dati sanitari. L'ovvia conseguenza di questa situazione è chiaramente la presenza di troppe informazioni duplicate inserite da diversi sistemi proprietari che non si interfacciano tra di loro, complicando per il cittadino la possibilità di accesso e reperimento di tali dati. Inoltre, i dati sanitari vengono spesso trascritti manualmente, portando alla loro perdita e, in alcuni casi, trascrizione errata o incompleta.\newline
Tra i vari servizi sanitari in grado di gestire queste informazioni è possibile considerare: il Fascicolo Sanitario Elettronico accessibile attraverso il portale della regione di residenza che riporta la storia sanitaria del cittadino e che risulta popolato dalle strutture sanitarie o dalle figure professionali presso le quali la persona si è recata; la Scheda Sanitaria Individuale compilata dal Medico di Medicina Generale che presenta i dati del paziente e tiene traccia della storia clinica dello stesso che tuttavia rimane solamente accessibile dalla figura che lo ha popolato; l'applicazione ICE che permette di contattare i soccorsi sanitari in caso di emergenza e che è installabile sui propri dispositivi; la Carta d'Identità Salvavita e i relativi documenti correlati sviluppati da Medici Volontari Italiani che offrono al cittadino documenti di utilità da portare sempre con sé in caso di necessità.\newline

In una situazione come quella attuale l'obiettivo fondamentale è cercare di integrare quanto più possibile questi diversi sistemi per realizzare un prodotto accessibile dall'utente finale e manutenibile dalle stesse entità senza la presenza di informazioni duplicate o contrastanti. Per un settore estremamente importante come quello sanitario la presenza di un unico sistema in grado di avere l'accesso immediato all'inserimento, alla modifica e all'utilizzo di tali dati risulta necessario.

\begin{figure}[H]
\centering
\includegraphics[scale = 0.4]{statoarte}
\caption{Stato dell'arte}
\end{figure}

\subsection{Carta d'Identità Salvavita}
Medici Volontari Italiani ha sviluppato, con l'aiuto di Fondazione IBM, Società Gisette e il Comune di Milano, un sistema in grado di digitalizzare i dati sanitari salvavita in modo che siano utilizzabili dal soccorritore in caso di necessità. Tramite l'inserimento delle informazioni del cittadino, il sistema genera la Carta d'Identità Salvavita, una tessera cartacea che presenta sulla facciata sinistra i dati anagrafici e i dati salvavita forniti dall'MMG e sulla parte destra il cosiddetto badge, ovvero un QR code riassuntivo di tali informazioni e i numeri di emergenza da contattare in caso di necessità. L'obiettivo fondamentale di tale operazione è la possibilità di avere sempre con sé un documento in grado di fornire, in situazioni di emergenza, informazioni fondamentali ed estremamente importante per il soccorritore.

\begin{figure}[H]
\centering
\includegraphics[scale = 0.2]{cis}
\caption{La Carta d'Identità Salvavita}
\end{figure}

\subsection{Braccialetto Salvavita}
Sviluppato parallelamente alla CIS, il braccialetto salvavita offre la possibilità di avere sempre a portata di mano il QR code la cui lettura comunica ai soccorritori le informazioni sanitarie d'emergenza definite precedentemente e segnala che la persona che lo indossa è dotata di una CIS. L'iniziativa è stata lanciata nel 2020 dal Municipio 3 dotando alcuni degli anziani residenti del braccialetto. Come mostrato nell'immagine seguente all'interno del braccialetto figurano i loghi delle società e delle entità che hanno collaborato alla sua creazione e soprattutto il codice QR di cui sopra.
\begin{figure}[H]
\centering
\includegraphics[scale = 0.4]{braccialetto}
\caption{Il braccialetto}
\end{figure}

\section{Obiettivo del Progetto}
Partendo dalla presenza di diverse realtà differenti che si occupano di fornire al cittadino l'obiettivo comune di salvaguardare la sua salute, questo progetto nasce con la necessità di poter creare un punto di partenza per realizzare un sistema in grado di unificare, per quanto possibile, il flusso di informazioni appena definito cercando di semplificare l'inserimento dei dati, la loro gestione e  soprattutto l'accesso e l'utilizzo da parte dei cittadini attraverso la generazione di documenti di utilità.
Il sistema realizzato salva i dati sanitari del cittadino attraverso l'applicazione web a cui hanno accesso l'MMG e il volontario, attraverso un form che viene compilato dagli stessi includendo tutte le informazioni necessarie. I dati del cittadino vengono salvati nel database e resi accessibili all'utente finale attraverso l'applicazione mobile tramite la quale è possibile avere tali dati sempre a portata di mano e generare e stampare la CIS, il badge e altri documenti di utilità per essere sempre in grado di fornirli al soccorritore in caso di necessità. In questo modo l'MMG è in grado di gestire e modificare i dati relativi ai suoi pazienti e i cittadini sono in grado di avere l'accesso immediato a tali dati.

\subsection{Obiettivi}
Gli obiettivi (goals) fondamentali dell'applicazione sono enunciati in seguito evidenziandone la suddivisione in 3 categorie principali per differenziare le funzionalità in base alla tipologia di utente per la quale sono state pensate. Gli obiettivi \textbf{Generali} sono quelli che accomunano tutte le tipologie di utenti dell'applicazione, tra cui possono figurare i cittadini, i volontari o anche i soccorritori che non hanno la necessità di creare un account. Vengono poi mostrati i goals più specifici che sono pensati per il \textbf{Cittadino} e per il \textbf{Volontario} in quanto rappresentano una parte dell'applicazione accessibile da solo una delle due tipologie. 

\subsubsection{Generali}
\paragraph{[G1]} Permettere all'utente di eseguire il reset della password tramite l'invio via mail di un link;
\paragraph{[G2]} Mostrare una lista di numeri di emergenza che possono essere contattati in caso di necessità;
\paragraph{[G3]} Permettere all'utente di eseguire la scansione di un codice QR e mostrare i relativi dati.

\subsubsection{Cittadino}
\paragraph{[G4]} Generare partendo dai dati del cittadino il QR code, il Profilo Sanitario Sintetico, la Carta d'Identità Salvavita, il badge e il braccialetto;
\paragraph{[G5]} Permettere al cittadino di diventare utente dell'applicazione a seguito dell'inserimento dei suoi dati da parte del MMG e del volontario;
\paragraph{[G6]} Permettere all'utente di eseguire il login nel sistema attraverso l'utilizzo di valide credenziali (email e password);
\paragraph{[G7]} Mostrare all'utente il QR code che riassume le informazioni salvavita con la possibilità di aprirlo come immagine, salvarlo nel dispositivo o stamparlo in formato .pdf;
\paragraph{[G8]} Mostrare all'utente i documenti fondamentali generati dai suoi dati sanitari, quali il Profilo Sanitario Sintetico, la Carta d'Identità Salvavita, il badge e il braccialetto;
\paragraph{[G9]} Mostrare lo storico Covid dell'utente con la possibilità di aprire i documenti relativi sul web e il QR code riassuntivo degli stessi con la possibilità di aprirlo come immagine, salvarlo nel dispositivo o stamparlo in formato .pdf;
\paragraph{[G10]} Permettere all'utente di contattare l'MMG e il volontario incaricato dell'inserimento e della gestione dei dati tramite email o numero telefonico.
\paragraph{[G11]} Permettere al cittadino di selezionare i dati precedenti a quelli attuali e i relativi documenti in base alla data.

\subsubsection{Volontario}
\paragraph{[G12]} Permettere al volontario di eseguire il login nel sistema attraverso l'utilizzo di valide credenziali (email e password);
\paragraph{[G13]} Permettere al volontario di ricercare tra i cittadini a lui assegnati e selezionarli;
\paragraph{[G14]} Permettere al volontario di selezionare per il singolo cittadino i dati scelti in base alla data in cui sono stati inseriti;
\paragraph{[G15]} Permettere al volontario di stampare la Carta d'Identità Salvavita, il badge e il braccialetto di un cittadino;
\paragraph{[G16]} Permettere al volontario di stampare la Carta d'Identità Salvavita, il badge e il braccialetto a più cittadini nello stesso momento.


\chapter{Panoramica del progetto}

\section{Architettura}
L'intero sistema può essere suddiviso in 3 parti fondamentali. Ognuna di esse si occupa di un utilizzo differente per quanto concerne i dati del cittadino, potendo di conseguenza sviluppare indipendentemente le nuove funzionalità, ma al tempo stesso aumentare l'integrazione tra le stesse.\newline
L'\textbf{applicazione web}, sviluppata da Davide Laffi, è realizzata con lo scopo di essere utilizzata dall'MMG o dal Volontario per la fase di inserimento dati sanitari e salvavita relativi al Cittadino.\newline
L'\textbf{applicazione mobile}, sviluppata da me, è realizzata con lo scopo di essere utilizzata dal Cittadino per avere traccia delle sue informazioni sanitarie e salvavita e di conseguenza accedere ai documenti generati e dal Volontario che ha la facoltà di ricercare uno o più utenti con lo scopo di fornire servizi di utilità ai cittadini, quali la stampa dei loro documenti.\newline
Il \textbf{lato server}, infine, è realizzato utilizzando Google Firebase, e fornisce il servizio di autenticazione per gli utenti e il servizio di database che consente il salvataggio delle informazioni relative ai cittadini.\newline
Nelle sezioni seguenti verranno analizzate le 3 parti nel dettaglio, focalizzando l'attenzione sulle tecnologie utilizzate.\newline

\begin{figure}[H]
\centering
\includegraphics[scale = 0.4]{architecture}
\caption{Architettura}
\end{figure}

\section{Web Application}
L'applicazione web è stata sviluppata da Davide Laffi. Questa parte sarà descritta a seguito del lavoro di Davide.

\section{Mobile Application}
L'applicazione mobile è stata realizzata utilizzando Flutter. La scelta di tale framework è avvenuta per la possibilità di sviluppare con un singolo linguaggio un'applicazione cross-platform accessibile da diversi sistemi operativi e dispositivi. L'applicazione è stata quindi organizzata in packages per poter disaccoppiare quanto più possibile le varie classi aumentando in questo modo l'ordine e la manutenzione del sistema, con la possibilità di aggiungere elementi in uno sviluppo futuro senza la necessità di dover ridisegnare da capo la struttura. Ciò che infatti caratterizza l'applicazione e più che altro la programmazione e la progettazione in Flutter è la possibilità di fare uso di widgets personalizzati per implementare funzionalità ripetibili in varie parti del sistema e riutilizzabili o modificabili in base alle proprie necessità. In questo modo le schermate che si vanno a costruire sono formate da elementi atomici la cui modifica non va ad intaccare o ostacolare le componenti vicine.

\subsection{Flutter}
\textbf{Flutter} è un UI software development kit open-source creato da Google e rilasciato nel 2017 e utilizzabile per sviluppare applicazioni mobile cross-platform. Il codice di Flutter è scritto in Dart, un linguaggio di programmazione client-optimized per applicazioni su multiple piattaforme, sviluppato anch'esso da Google e utilizzato per l'implementazione di applicazioni mobile, desktop, server e web. Tra le caratteristiche principali di Flutter è bene considerare sicuramente la natura \textbf{cross-platform} che, con permette di sviluppare un singola applicazione su Android, iOS e sul web senza la necessità di scrivere codice specifico. Lo stesso Flutter si integra alla perfezione con \textbf{Firebase}, in quanto anch'esso sviluppato da Google, e permette di avere l'accesso a diverse funzionalità utili rigurdanti la gestione dei dati e l'autenticazione degli utenti. La scrittura dell'applicazione è semplificata dalla presenza dei \textbf{widgets}, componenti atomici che rappresentano elementi della UI e che possono essere uniti nella creazione di schermate pulite e complesse allo stesso tempo. Inoltre esistono svariati \textbf{plugins} e \textbf{packages} già pronti all'uso che permettono di arricchire il sistema con funzionalità e servizi aggiuntivi senza la necessità di scriverli partendo da zero.

\section{Lato Server}
Il lato server del sistema è stato realizzato utilizzando \textbf{Google Firebase}, una piattaforma online che permette di salvare, sincronizzare e utilizzare i dati generati dalle applicazioni web e mobile. Dalla moltitudine di funzionalità offerte dal servizio, per lo sviluppo del sistema sono state sfruttate in particolare due di esse: il servizio di \textbf{autenticazione} abilita la creazione e la gestione degli utenti per il sistema. Tra i vari metodi di sign-in disponibili è stato scelto quello classico di iscrizione e login tramite email e password. Nel servizio di autenticazione è fornita anche la possibilità del reset della password; per quanto concerne il database dell'applicazione è stato sfruttato \textbf{Firestore Database} che permette la creazione di collezioni e documenti che andranno a popolare i dati che verranno utilizzati dal sistema. Il vantaggio fondamentale nell'utilizzo di Firebase consiste nella sua perfetta integrazione con Flutter in quanto sono entrambi servizi Google.

\subsection{Autenticazione}
Poichè l'intero sistema si basa su 3 profili utente principali (MMG, volontario e cittadino), risulta necessario un servizio di autenticazione efficiente in grado servire questo scopo. In particolare l'accesso avviene tramite email e password, considerato il metodo di autenticazione più consono per la tipologia di sistema sviluppato. In questo modo, infatti, i volontari incaricati hanno la possibilità di creare gli account relativi ai cittadini a seguito della fase di inserimento dati avvenuta per mano dell'MMG e del volontario stesso. In questo modo viene inviata agli utenti una mail contenente la password per l'account generata in maniera randomica, permettendo un accesso veloce e semplice anche alle persone meno pratiche nella creazione di account. Inoltre, con questa tipologia di credenziali è possibile al tempo stesso garantire l'accesso solamente alle entità professionali indicate e favorire gli utenti senza account social provvedendo ad un metodo di autenticazione che sfrutta gli indirizzi email utilizzati ormai dalla gran parte della popolazione.\newline

In generale i metodi di Firebase utilizzati in fase di autenticazione sono essenzialmente i seguenti:
\begin{itemize}
\item \texttt{signInWithEmailAndPassword}: consente di eseguire il login di un utente attraverso email e password nel caso in cui le credenziali fossero valide;
\item \texttt{sendPasswordResetEmail}: permette di inviare un link per il reset della password all'indirizzo email specificato.
\end{itemize}

\subsection{Firestore Database}
Cloud Firestore è un database cloud NoSQL che permette di salvare e utilizzare i dati lato client in maniera istantanea e rapida. I dati vengono sincronizzati in tempo reale e viene offerto supporto per il lavoro indipendentemente dalla latenza di rete. Le principali funzionalità offerte, reperibili nella documentazione ufficiale, sono la flessibilità, l'interrogazione espressiva e il supporto offline. I dati sono infatti organizzati in collezioni e documenti che permettono un accesso veloce attraverso queries specifiche o generiche che possono utilizzare filtri e criteri di ordinamento. Le informazioni presenti nel database vengono memorizzate automaticamente nella cache dell'applicazione in modo che il sistema sia in grado di funzionare anche in assenza di rete, con la possibilità di aggiornare istantaneamente tali dati al recupero della connessione.\newline

Il database del sistema è stato organizzato in 4 collezioni fondamentali per considerare tutti i possibili attori dello stesso. Nel seguito verranno analizzate nel dettaglio focalizzandosi sulle caratteristiche dei documenti che contengono. Il seguente diagramma ER offre una panoramica dettagliata della composizione del database e permette di comprendere i legami tra le 4 collezioni.\newline

\textbf{IMMAGINE SCHEMATICA}

\subsubsection{Users}
La collezione \textbf{users} è pensata con lo scopo di tenere traccia dei dati fondamentali degli utenti che hanno accesso al sistema con la possibilità di differenziarli in base alla tipologia di account. La collezione, infatti, contiene un documento per ogni utente, caratterizzato da un id univoco assegnato in fase di creazione dell'account da Firebase che permette di tenere anche traccia della connessione. All'interno del documento sono specificati i dati relativi all'account tra cui l'id, il codice fiscale necessario per collegare un profilo utente ai dati inseriti relativi ai cittadini, l'email e la tipologia di account che permette di differenziare i cittadini, dagli MMG e dai volontari.

\subsubsection{Patients}
La collezione \textbf{patients} include tutti i dati relativi ai cittadini. Ogni documento della collezione è reso unico dal codice fiscale relativo e contiene una sottocollezione che permette di avere l'accesso allo storico dei dati del cittadino stesso. Infatti, grazie a questa organizzazione, l'utente ha accesso alle sue informazioni sanitarie più recenti, ma può, all'occorrenza avere traccia di quelle passate, selezionabili nell'applicazione attraverso la data in cui sono state inserite.\newline

Ogni documento di tale collezione è ulteriormente collegato da un campo al documento relativo al medico di medicina generale e al volontario incaricati dell'inserimento dei dati permettendo da un lato di contattarli e dall'altro di avere l'accesso, per quanto riguarda le due figure professionali appena citate, alla collezione completa di cittadini a loro assegnati.

\subsubsection{Doctors}
La collezione \textbf{doctors} contiene i documenti relativi agli MMG che utilizzano il sistema per l'inserimento e la modifica dei dati sanitari relativi ai propri pazienti. Ogni documento, analogamente a quanto succede con le altre categorie, è reso univoco attraverso il codice fiscale del medico e contiene i dati anagrafici relativi.

\subsubsection{Volunteers}
La collezione \textbf{volunteers} contiene i documenti relativi ai volontari incaricati della gestione dei dati e dell'evenutale stampa di documenti dei cittadini. Ogni elemento della collezione è caratterizzato dal codice fiscale del volontario e contiene i dati anagrafici dello stesso. Attraverso il codice fiscale, il volontario ha accesso all'insieme dei cittadini di sua competenza per i quali è in grado di fornire supporto.

\chapter{Implementazione}
\section{Packages e Classi}
L'applicazione mobile è stata sviluppata utilizzando Flutter 2.0. Le varie classi dell'applicazione sono state suddivise in packages in base alla tipologia per incrementare la manutenibilità, la chiarezza e la separazione delle stesse Il sistema viene di conseguenza suddiviso in 4 macro-aree ognuna delle quali rappresenta una parte fondamentale del funzionamento. Con questo tipo di suddivisione sarà possibile in futuro incrementare le funzionalità dell'applicazione senza intaccare la struttura generale ma andando ad arricchire quanto già presente.\newline

\textbf{IMMAGINE SCHEMATICA}\newline

Di seguito verranno analizzati i 4 packages di cui si compone l'applicazione focalizzandosi, se necessario, sui singoli metodi della classi considerate.

\subsection{Model}
\begin{figure}[H]
\centering
\includegraphics[scale = 1.0]{model}
\caption{The Model files}
\end{figure}
Il package Model contiene i vari oggetti utilizzati all'interno dell'applicazione che spesso si presentano con la presenza dei vari campi, del costruttore e di metodi di utilità generale per la classe. In particolare:
\begin{itemize}
\item \texttt{end\_user}: rappresenta l'utente finale dell'applicazione e permette di salvare le sue informazioni fondamentali ottenute da Firebase quali l'id, il codice fiscale, l'email e la tipologia di utente (cittadino o volontario);
\item \texttt{patient}: rappresenta il cittadino e contiene le informazioni invariabili (codice fiscale, nome e cognome) e una mappa di \texttt{TimestampPatient} organizzata per data con la funzione di storico per i dati precedenti a quelli attuali;
\item \texttt{searched\_patient}: rappresenta la classe utilizzata per gestire la ricerca dei cittadini da parte del volontario;
\item \texttt{timestamp\_patient}: contiene tutte le informazioni sanitarie e salvavita del cittadino relative ad una certa data e metodi di utilità per accedere ai dati.
\end{itemize}

\subsection{Screens}
\textbf{IMMAGINE SCHEMATICA}\newline

Il package Screens contiene le schermate principali dell'applicazione. Ogni file corrisponde ad una schermata i cui widgets sono spesso contenuti nel package Widgets. In particolare:
\begin{itemize}
\item \texttt{emergency\_numbers}: rappresenta la schermata dei Numeri Utili e si suddivide in due tabs fondamentali. La prima tab contiene i contatti (email e numero di telefono) dell'MMG e del volontario ed è costruita utilizzando le \texttt{NumbersCard}. La seconda tab contiene i numeri di emergenza da contattare in caso di necessità ed è costruita usando le \texttt{EmergencyNumberTile};
\item \texttt{homepage}: rappresenta la schermata principale dell'applicazione per quanto riguarda il cittadino. Si suddivide in 3 tab fondamentali, ovvero:
\begin{itemize}
\item QR Code: composta dal codice QR relativo alle informazioni salvavita e ai 3 \texttt{FunctionButton};
\item Informazioni: composta da 4 \texttt{FunctionCard}, ognuna relativa a un documento del cittadino;
\item Covid19: composta dal codice QR relativo allo storico Covid e ad una lista di \texttt{CovidTile}.
\end{itemize}
\item \texttt{login}: rappresenta la schermata in cui gli utenti possono effettuare l'accesso all'applicazione. È realizzata utilizzando il package \texttt{flutter\_login} che garantisce la presenza di una schermata pulita e fornita di tutte le funzionalità necessarie al suo utilizzo;
\item \texttt{qr\_code\_scanner}: rappresenta la schermata responsabile della scansione dei codici QR. Si compone di un \texttt{FunctionButton} che permette di avviare lo scanner e, in caso di scansione corretta, presenta i dati letti dal codice QR;
\item \texttt{volunteer}: rappresenta la Schermata del Volontario e si compone di una barra di ricerca realizzata attraverso il package \texttt{material\_floating\_search\_bar}. Gli utenti selezionati vengono inseriti in una lista di \texttt{VolunteerCard};
\item \texttt{wrapper}: rappresenta il widget responsabile del reindirizzamento degli utenti nella schermata corretta. Se l'utente risulta già loggato dalla sessione precedente, allora lo indirizza alla pagina relativa, altrimenti mostra la schermata di Login e attende le credenziali per reindirizzare l'utente nella Homepage (cittadino) o nella Schermata del Volontario (volontario).
\end{itemize}

\subsection{Services}
\textbf{IMMAGINE SCHEMATICA}\newline

Il package Services contiene le classi responsabili dei principali servizi utilizzati nelle varie schermate e nella logica dell'applicazione. In particolare:
\begin{itemize}
\item \texttt{auth\_service}: rappresenta la classe incaricata di gestire l'autenticazione degli utenti. Si compone dei seguenti metodi:
\begin{itemize}
\item \texttt{getCurrentUser}: restituisce l'utente Firebase corrente;
\item \texttt{login}: esegue il login dell'utente specificato attraverso email e password e restituisce un booleano rappresentante il risultato dell'operazione;
\item \texttt{resetPassword}: permette di inviare all'utente il link per il reset della password alla mail specificata.
\end{itemize}
\item \texttt{database\_service}: è la classe che si occupa della gestione del database Firebase. Ha una variabile per ogni collezione (users, patients, volunteers) e i seguenti metodi:
\begin{itemize}
\item \texttt{getUser}: restituisce il documento Firebase relativo all'utente corrente. Viene mappato nella classe \texttt{EndUser} attraverso il metodo \texttt{\_userFromFirebase};
\item \texttt{getPatient}: restituisce il documento Firebase relativo al cittadino corrente insieme ai dati sanitari e salvavita. Viene mappato nella classe \texttt{Patient} attraverso il metodo \texttt{\_patientFromFirebase};
\item \texttt{getPatientsList}: restituisce la lista dei documenti Firebase relativi a tutti i cittadini di competenza del volontario specificato.Viene mappato nella lista di \texttt{Patient} attraverso il metodo \texttt{populatePatientsData}.
\end{itemize}
\item \texttt{pdf\_handler}: è la classe che si occupa della gestione dei file in formato .pdf all'interno dell'applicazione. In particolare per ogni documento viene specificato un metodo di creazione per il file. Successivamente per ogni file è presente un metodo che si occupi di:
\begin{itemize}
\item aprire il file;
\item scaricare il file;
\item condividere il file;
\item stampare il file.
\end{itemize}
\item \texttt{qr\_code\_handler}: è la classe che si occupa della gestione dei codici QR all'interno dell'applicazione. In particolare:
\begin{itemize}
\item \texttt{generateQRCode}: genera il codice QR partendo da una stringa di dati;
\item \texttt{openQRCode}: apre il codice QR dopo averlo generato;
\item \texttt{saveQRCodeToGallery}: salva nella galleria del dispositivo il codice QR in formato immagine;
\end{itemize}
\end{itemize}

\subsection{Widgets}
\textbf{IMMAGINE SCHEMATICA}\newline

Il package Widgets contiene i widgets utilizzati frequentemente nelle altre classi dell'applicazione con l'obiettivo di ridurre al minimo la duplicazione del codice. In particolare:
\begin{itemize}
\item \texttt{appbar\_button}: rappresenta le icone cliccabili nell'appbar a cui è collegata una funzione;
\item \texttt{covid\_tile}: rappresenta le tiles presenti nello storico Covid, caratterizzate dall'evento Covid, dalla data in cui è avvenuto e dal link al documento correlato;
\item \texttt{emergency\_numbers\_tile}: rappresenta le tiles presente nella tab Emergenza dei Numeri Utili, caratterizzate da un'icona, la descrizione del contatto e il numero telefonico;
\item \texttt{form\_text\_field}:
\item \texttt{function\_button}:
\item \texttt{function\_card}:
\item \texttt{function\_icon}:
\item \texttt{numbers\_card}: rappresenta le cards presenti nella tab Contatti dei Numeri Utili, caratterizzate da un'icona, il nome, la professione (medico o volontario) e i contatti (email e numero di telefono);
\item \texttt{processing\_indicator}: rappresenta l'indicatore di caricamento presente nelle schermate dell'applicazione;
\item \texttt{radio\_tile}:
\item \texttt{volunteer\_card}: rappresenta le cards relative ai cittadini presenti nella Schermata del Volontario. Ognuna di esse si compone del nome del cittadino, il codice fiscale e i pulsanti per la stampa dei documenti relativi.
\end{itemize}

\section{External Services and Libraries}
Durante la fase di sviluppo dell'applicazione mobile è stato fatto largo utilizzo dei packages forniti dal sito \texttt{pub.dev}. Le motivazioni alla base di questa scelta sono dovute alla possibilità di sfruttare la presenza di funzionalità già esistenti per arricchire l'applicazione evitando il riutilizzo di codice e favorendo la pulizia dello stesso. Nell'elenco seguente verranno menzionati tutti i packages utilizzati con le relative funzionalità principali:
\begin{itemize}
\item \texttt{firebase\_core}: per connettere l'applicazione ai vari servizi offerti da Firebase;
\item \texttt{firebase\_auth}: per utilizzare le funzionalità di autenticazione offerte da Firebase;
\item \texttt{cloud\_firestore}: per utilizzare le funzionalità del database cloud offerto da Firebase;
\item \texttt{path\_provider}: per ottenere percorsi comuni di cartelle nei dispositivi in cui viene eseguita l'applicazione;
\item \texttt{pdf}: per generare i file .pdf utilizzati nell'applicazione e modificarli;
\item \texttt{qr\_flutter}: per generare i codici QR utilizzati per i dati salvavita e per lo storico Covid;
\item \texttt{image\_gallery\_saver}: per salvare nella galleria del dispositivo i QR code;
\item \texttt{fluttertoast}: per mostrare messaggi in rilievo a seguito di operazioni eseguite con successo;
\item \texttt{open\_file}: per aprire i file .pdf in maniera nativa nel sistema di appartenenza;
\item \texttt{downloads\_path\_provider}: per ottenere il path relativo alla cartella di Download nei dispositivi mobili;
\item \texttt{permission\_handler}: per gestire i permessi all'interno dell'applicazione;
\item \texttt{flutter\_barcode\_scanner}: per fornire la funzionalità di scansione dei codici QR;
\item \texttt{url\_launcher}: per aprire pagine internet, indirizzi email o numeri telefoni con l'applicazione di sistema correlata;
\item \texttt{humanitarian\_icons}: per avere icone aggiuntive da utilizzare nella schermata relativa ai contatti di emergenza;
\item \texttt{printing}: per abilitare le funzionalità di condivisione e stampa dei vari documenti;
\item \texttt{path}: per eseguire operazioni di manipolazione sui percorsi del sistema in uso;
\item \texttt{universal\_html}: per gestire in maniera cross-platform il package html;
\item \texttt{material\_floating\_search\_bar}: per generare una barra di ricerca consona al Material design;
\item \texttt{flutter\_login}: per generare una schermata di login pulita ed efficiente;
\item \texttt{unicorndial}: per utilizzare un floating action button con molteplici funzionalità.
\end{itemize}


\chapter{Funzionalità}
\section{Panoramica dell'Applicazione}
L'applicazione è composta da una schermata principale e da un'insieme di pagine che forniscono diverse funzionalità. In particolare l'homepage è composta da 3 tabs, ognuna delle quali è stata pensata per uno scopo preciso. La prima, che rappresenta anche la schermata principale su cui si apre l'applicazione e la funzionalità fondamentale della stessa, mostra il QR code contentente le informazioni salvavita del cittadino con la possiblità di mostrarlo istantaneamente al soccorritore in caso di necessità. Scorrendo verso destra la schermata successiva mostra i documenti originali generati dai dati sanitari e dalle informazioni salvavita del cittadino che vengono creati sfruttando le informazioni inserite dall'MMG e dal volontario. Infine l'ultima tab mostra uno storico relativo agli ultimi eventi relativi all'argomento Covid del cittadino (come ad esempio i tamponi o le vaccinazioni effettuati) e un QR code che riassume tutti questi eventi. Entrambi i QR code e tutti i documenti presenti nell'applicazione possono essere aperti, salvati nel dispositivo o stampati.\newline

Oltre alla schermata principale sono presenti ulteriori funzionalità che offrono all'utente un sistema completo in grado di gestire i suoi dati e tutti i possibili servizi utili correlati ad essi. Il cittadino è infatti in grado di accedere al numero telefonico e all'indirizzo email relativi al volontario incaricato e al medico di medicina generale per contattarli in caso di necessità. In questa ottica si inserisce anche una piccola rubrica interna contenente tutti i numeri di emergenza che possono risultare utili come ad esempio quelli relativi alle forze dell'ordine o al primo soccorso. I soccorritori invece possono utilizzare l'applicazione senza la necessità di un account potendo utilizzare la funzionalità di scansione per analizzare all'occorrenza il codice QR di un paziente per avere accesso immediato ai dati salvavita dello stesso.\newline

Oltre alle funzionalità appena descritte è presente un'intera parte dell'applicazione riservata al volontario con lo scopo di abilitare la ricerca di uno o più cittadini con la facoltà di stampare per essi i documenti generati dai loro dati.

\section{Descrizione delle Funzionalità}
Nelle sezioni successive verranno analizzate nel dettaglio le funzionalità introdotte nel paragrafo precedente cercando di mettere in luce gli scopi e i modi d'utilizzo pensati. I paragrafi sono organizzati considerando nell'ordine ogni funzionalità e la relativa schermata in grado di fornirla.

\subsection{Registrazione}
La registrazione degli account avviene in due modi differenti in base alla tipologia di utente. La creazione dell'account del volontario (e anche quello dell'MMG) avviene per mano dell'amministratore di sistema a seguito di una richiesta. In questo modo è possibile assicurare solamente alle persone corrette il privilegio di esercitare la propria professione e avere l'accesso ai dati sensibili dei cittadini. L'account del cittadino, invece, viene creato dal volontario a seguito dell'inserimento dei dati sanitari da parte dell'MMG. In questo modo il cittadino può avere accesso all'applicazione solo a seguito dell'inserimento dei suoi dati.\newline
Le password degli accounts sono generate in maniera randomica e inoltrate agli utenti via mail. A seguito di ciò l'utente sarà libero di modificarla secondo le proprie preferenze.

\subsection{Login}
La schermata di Login permette agli utenti di accedere al sistema fornendo credenziali valide e fornite a seguito del processo di registrazione analizzato precedentemente. Gli utenti sono in grado di modificare la propria password cliccando su "Hai dimenticato la password?" che invierà all'utente una mail contenente un link per il reset della stessa. Sia il cittadino che il volontario condividono la stessa scheramta di Login. Il sistema controlla le credenziali e reindirizza l'utente alla schermata corretta.\newline

Tramite la schermata di Login è possibile accedere, senza necessità di avere un account o di aver eseguito l'accesso, a due ulteriori schermate che verranno analizzate dettagliatamente in seguito, ovvero: la schermata dei \textbf{Numeri Utili}, che permette di utilizzare questa componente dell'applicazione come una rubrica formata da tutti i numeri di emergenza accessibili in caso di necessità;
la schermata chiamata \textbf{Scanner QR Code}, il cui accesso è stato pensato per i soccorritori, unici veri utilizzatori di questo servizio, con l'obiettivo di velocizzare il processo di scansione il più possibile senza la necessità di possedere necessariamente un account.

\subsection{Schermata del Volontario}
Il volontario ha accesso ad una singola schermata nell'applicazione mobile in quanto la maggior parte delle sue facoltà sono correlate all'utilizzo dell'applicazione web. Il volontario, tramite un'apposita barra di ricerca, è in grado di ricercare attraverso nome o codice fiscale gli utenti di sua competenza e selezionarli. A seguito della selezione è in grado di eseguire la stampa dei documenti fondamentali (CIS, badge e braccialetto) del singolo o del gruppo di utenti selezionati. Per ogni utente, il volontario può modificare la data dei documenti scegliendo di stampare quelli richiesti.

\subsection{Homepage}
L'Homepage è la schermata principale dell'applicazione a cui ha accesso il cittadino. Questa schermata è suddivisa in 3 tabs principali che forniscono diverse funzionalità:
\begin{enumerate}
\item la prima tab mostra il QR code salvavita del cittadino contenente:
\begin{itemize}
\item Nome e cognome;
\item Data di nascita;
\item Gruppo sanguigno;
\item Primo contatto ICE;
\item Secondo contatto ICE;
\item Lista delle eventuali patologie;
\item Lista delle eventuali allergie;
\item Informazioni aggiuntive.
\end{itemize}
e permette all'utente di aprirlo come immagine, salvarlo nella galleria del dispositivo o stamparlo in formato .pdf;
\item la seconda tab mostra i documenti principali generati dai dati sanitari e salvavita inseriti dall'MMG e dal volontario incaricato:
\begin{itemize}
\item Profilo Sanitario Sintetico: contiene tutti i dati sanitari e salvavita relativi al cittadino;
\item Carta d'Identità Salvavita: mostra la CIS del cittadino contenente la foto, il QR code, i contatti ICE e le informazioni salvavita;
\item Badge: mostra la foto, il QR code e i contatti ICE;
\item Braccialetto: contiene i loghi delle società partner e il QR code.
\end{itemize}
\item la terza tab mostra gli eventi relativi alla storia Covid del cittadino, come ad esempio i tamponi effettuati o le vaccinazioni con la possibilità di aprire i documenti relativi inseriti dall'MMG o dal volontario. La schermata contiene anche un QR code riassuntivo dei vari eventi.
\end{enumerate}
Tramite l'Homepage il cittadino è in grado di accedere alle altre funzionalità dell'applicazione (Numeri Utili e Scanner QR Code) o di disconnettersi. Il cittadino può anche selezionare la data relativa ai suoi dati scegliendoli in base al periodo in cui sono stati aggiunti o modificati dall'MMG.

\subsection{Numeri Utili}
La schermata dei Numeri Utili è accessibile sia dalla schermata di Login, sia dall'Homepage e presenta due funzionalità principali. L'utente, senza necessità di eseguire il login o di avere un account esistente, può avere accesso diretto ad una piccola rubrica contenente numeri di emergenza che possono essere contattati in caso di necessità, ovvero:
\begin{itemize}
\item carabinieri;
\item polizia di stato;
\item vigili del fuoco;
\item guarda di finanza;
\item emergenza sanitaria.
\end{itemize}
Se il cittadino ha eseguito il login all'applicazione, nella stessa schermata, può usufruire di un ulteriore funzionalità che gli permette di contattare l'MMG e il volontario incaricato della gestione dei suoi dati tramite email o numero telefonico.

\subsection{Scanner QR Code}
La schermata dello Scanner è accessibile sia dalla schermata di Login, sia dall'Homepage. Questa funzionalità è stata pensata per essere utilizzata dai soccorritori che, senza dover necessariamente fare l'accesso all'applicazione, possono utilizzare lo scanner QR per avere accesso immediato alle informazioni salvavita dei cittadini in difficoltà. L'applicazione infatti, a seguito della scansione, mostra i dati del cittadino in forma leggibile ed organizzata.

\chapter{Interfaccia Utente e Casi d'Uso}
\section{UI Design}
Il design dell'interfaccia utente è basato principalmente sulle linee guida del Material design, cercando di focalizzarsi simultaneamente sulla semplicità d'uso e sull'efficienza delle funzionalità offerte. Attraverso schermate pulite e prive di troppi elementi a schermo, per l'utente risulta chiaro istantaneamente lo scopo di ogni singola componente. Le funzionalità dell'applicazione risultano atomiche e, oltre a non rallentare l'utilizzo del sistema, permettono di raggiungere immediatamente lo scopo per cui sono state pensate. Nei prossimi paragrafi verranno analizzate in dettaglio le scelte stilistiche e progettuali volte a soddisfare i requisiti appena definiti, focalizzandosi sulle motivazioni di tali scelte e mostrando le schermate fondamentali dell'applicazione.

\subsection{Logo}
\subsection{Schermate Principali}
L'applicazione si compone di poche schermate fondamentali con lo scopo di limitare al minimo la confusione per l'utente finale, offrendo le funzionalità richieste nella maniera più accessibile possibile. Di seguito, facendo riferimento a queste schermate, verrà analizzata la disposizione degli elementi sullo schermo evidenziandone le caratteristiche.

\subsubsection{Login}
\begin{figure}[H]
\centering
\includegraphics[scale = 0.2]{mobile}
\includegraphics[scale = 0.2]{mobile}
\caption{La schermata di Login}
\end{figure}
La schermata di Login presenta il titolo dell'applicazione e un singolo form in cui all'utente è richiesto di inserire le proprie credenziali, composte da indirizzo email e password, nei campi relativi. Tramite il bottone "Login" viene eseguito un controllo su tali dati, ne viene verificata l'esistenza e in caso positivo l'utente viene indirizzato nella schermata corretta in base alla tipologia di account (il volontario avrà accesso alla \textbf{Schermata del Volontario}, il cittadino avrà accesso alla \textbf{Homepage}). Nel caso in cui le credenziali fossero errate o non soddisfacessero i requisiti (ad esempio una password deve essere formata da un minimo di caratteri), la schermata mostrerebbe all'utente l'errore relativo.\newline
Al di sotto del bottone di "Login" figurano due icone che permettono di eseguire l'accesso alle schermate dei \textbf{Numeri Utili} e del \textbf{Scanner QR Code} senza la necessità di eseguire l'accesso con un account.\newline

Nel caso in cui l'utente avesso dimenticato la propria password sarebbe in grado di eseguire il reset della stessa cliccando sulla scritta "Password dimenticata?" che, mantenendo l'utente nella stessa schermata, mostra un form differente in cui inserire l'indirizzo email a seguito di cui è necessario cliccare sul bottone "Reset".

\subsubsection{Schermata del Volontario}
\begin{figure}[H]
\centering
\includegraphics[scale = 0.2]{mobile}
\includegraphics[scale = 0.2]{mobile}
\includegraphics[scale = 0.2]{mobile}
\caption{La schermata Login}
\end{figure}
La Schermata del Volontario è l'unica componente dell'applicazione pensata esclusivamente per questa tipologia di account. I compiti principali del volontario sono infatti svolti nel lato web dell'intero sistema e consistono nell'inserimento dei dati del cittadino. \newline
Tale schermata presenta nell'angolo destro l'icona necessaria per eseguire il logout dall'applicazione e nella parte alta una singola barra di ricerca. Tramite la stessa il volontario è in grado di eseguire la ricerca dei cittadini di sua competenza inserendone il nome o il codice fiscale. A seguito di una ricerca vengono mostrati tutti i risultati che soddisfano la query ed è possibile aggiungere tali cittadini alla lista attraverso il tasto "Aggiungi". A seguito di ciò la schermata viene aggiornata mostrando il cittadino appena selezionato con la possibilità di stamparne i documenti relativi (Badge, CIS e Braccialetto). Il riquadro del singolo utente mostra, oltre al nome, al codice fiscali e ai documenti sopra citati, la data relativa all'ultimo aggiornamento avvenuto in merito alle informazioni del cittadino. I dati possono essere modificati cliccando su una delle due icone presenti nella parte alta a destra del riquadro che mostra le date relative all'inserimento di tali dati. L'altra icona permette di rimuovere il cittadino dalla lista.

\subsubsection{Homepage}
L'\textbf{Homepage} dell'applicazione è la schermata più ricca di funzionalità in quanto offre tutti i servizi fondamentali per il cittadino.\newline
\begin{figure}[H]
\centering
\includegraphics[scale = 0.2]{mobile}
\caption{La schermata Homepage - Codice QR}
\end{figure}
La tab \textbf{Codice QR} mostra in primo piano il codice QR relativo ai dati salvavita del cittadino in modo che sia istantaneamente accessibile in caso di necessità. Sotto di esso sono presenti 3 bottoni, ovvero "Apri", "Salva", "Stampa", che permettono rispettivamente di aprire il QR code in un'altra schermata in formato immagine, salvarlo nel dispositivo e aprire la schermata nativa per abilitare la stampa dello stesso in formato .pdf.

\begin{figure}[H]
\centering
\includegraphics[scale = 0.2]{mobile}
\caption{La schermata Homepage - Informazioni}
\end{figure}
La tab \textbf{Informazioni} presenta 4 riquadri, ognuno dei quali relativo ad uno dei documenti di utilità generati dall'applicazione (Dati, Badge, CIS e Braccialetto). Ognuno dei riquadri presenti un'immagine rappresentativa, il nome del documento e una breve descrizione. Sono poi presenti 3 icone che nell'ordine permettono l'apertura del documento in formato .pdf, il salvataggio nel dispositivo e la stampa dello stesso.

\begin{figure}[H]
\centering
\includegraphics[scale = 0.2]{mobile}
\caption{La schermata Homepage - Covid19}
\end{figure}
La tab \textbf{Covid19} presenta nella parte alta della schermata una disposizione analoga degli elementi alla prima tab, in cui è presente il codice QR relativo al Covid del cittadino con la possibilità di eseguirne la scansione per mostrare a chi di dovere gli ultimi eventi relativi alla propria storia clinica relativa a ciò. Nella parte sottostante è presente una lista di elementi che sono formati dal nome dell'evento, dalla data e da un bottone "Apri" che permette di accedere al documento relativo.\newline

Nella parte alta della schermata di Homepage, comunemente a tutte e 3 le tabs sono presenti diverse icone. Nella parte sinistra, accanto al titolo è presente l'icona per eseguire la disconnessione. Nella parte destra, invece, figurano 3 icone che permettono rispettivamente di accedere alle schermate \textbf{Numeri Utili}, \textbf{Scanner QR Code} e di scegliere tra le date disponibili quella di interesse per mostrare i dati.

\subsubsection{Numeri Utili}
\begin{figure}[H]
\centering
\includegraphics[scale = 0.2]{mobile}
\includegraphics[scale = 0.2]{mobile}
\caption{La schermata Numeri Utili}
\end{figure}
La schermata dei Numeri Utili presenta nella parte alta due icone relative alle due tabs di cui si compone. La prima, quella relativa ai \textbf{Contatti} mostra due riquadri che contengono il numero telefono e l'indirizzo mail che permettono di contattare l'MMG o il volontario incaricato. La seconda tab, chiamata \textbf{Emergenza}, è accessibile anche dalla schermata di \textbf{Login} senza la necessità di aver eseguito l'accesso e presenta una lista di elementi in cui è presenta un'icona, il numero telefonico e il nome dell'entità a cui appartiene e che può essere contattata in caso di emergenza.

\subsubsection{Scanner QR Code}
\begin{figure}[H]
\centering
\includegraphics[scale = 0.2]{mobile}
\includegraphics[scale = 0.2]{mobile}
\includegraphics[scale = 0.2]{mobile}
\caption{La schermata Scanner QR Code}
\end{figure}
La schermata \textbf{Scanner QR Code} presenta inizialmente un singolo bottone "Scan". Dopo averlo cliccato viene utilizzata la camera del dispositivo per eseguire la scansione, con la possibilità di attivare il flash se necessario. A seguito di ciò, nel caso in cui il codice QR sia relativo ai dati salvavita di un cittadino, tali dati vengono mostrati ad elenco, altrimenti viene segnalato l'errore nella scansione.

\subsection{Layout Tablet}
L'applicazione, seppur sviluppata e pensata per dispositivi mobile in quanto le funzionalità della stessa si prestano maggiormente alla portabilità costante e al loro accesso immediato in qualunque situazione, è scaricabile, installabile e usufruibile anche da tablet. Sebbene fondamentalmente gli elementi presentati rimangano gli stessi, in alcune schermate la disposizione varia per sfruttare la tipologia di schermo più ampia e solitamente in orizzontale. Di seguito vengono presentate le schermate in cui è possibile notare questo tipo di cambiamenti.

\subsection{Layout Web}
L'applicazione, inoltre, grazie alla sua implementazione tramite Flutter, è accessibile anche dal web attraverso l'url: \texttt{https://psscis.web.app/}. Analogamente a quanto successo con i tablet, la disposizione degli elementi è stata ridisegnata in alcuni casi per sfruttare gli schermi con orientamento orizzontale. Inoltre, alcune schermate, come ad esempio la \textbf{Schermata del Volontario}, sono pensate per essere utilizzate su schermi di computer permettendo una visualizzazione migliore dei dati. Di seguito vengono presentate le schermate in cui è possibile notare questo tipo di cambiamenti.

\section{Use Cases}
Nelle sezioni successive verrano analizzati i casi d'uso più frequenti, avendo cura di specificare:
\begin{itemize}
\item Scenario: il titolo riassuntivo del caso d'uso analizzato;
\item Condizione d'ingresso: requisiti minimi per l'esecuzione del caso d'uso;
\item Flusso degli eventi: azioni o avvenimenti che caratterizzano il caso d'uso;
\item Condizione d'uscita: evento finale che termina il flusso degli eventi;
\item Eccezioni: possibili avvenimenti che causano l'interruzione o privano l'esecuzione degli eventi.
\end{itemize}


\subsection{Login}
\begin{table}[H]
\centering
\begin{tabular}{|p{4cm}|p{10cm}|}
\hline
Scenario & Login utente \\
\hline
Condizione d'ingresso & L'utente ha scaricato l'applicazione o ha eseguito l'accesso attraverso il web browser \newline
L'utente ha già un account creato dal volontario incaricato \\
\hline
Flusso degli eventi & 
\begin{enumerate}
\item L'utente inserisce la sua email nel campo "Email" della schermata
\item L'utente inserisce la sua password nel campo "Password" della schermata
\item L'utente clicca sul bottone di "Login"
\end{enumerate}\\
\hline
Condizione d'uscita & L'applicazione reindirizza l'utente nella schermata corretta, in base alla tipologia di account (cittadino o volontario)\\
\hline
Eccezioni & 
\begin{itemize}
\item L'utente non ha inserito credenziali valide
\item L'utente non ha inserito le credenziali in tutti i campi obbligatori
\item Il sistema non è in grado di completare la richiesta a causa di un errore interno
\end{itemize} \\
\hline
\end{tabular}
\end{table}

\begin{table}[H]
\centering
\begin{tabular}{|p{4cm}|p{10cm}|}
\hline
Scenario & Reset della password \\
\hline
Condizione d'ingresso & L'utente ha scaricato l'applicazione o ha eseguito l'accesso attraverso il web browser \newline
L'utente ha già un account creato dal volontario incaricato \\
\hline
Flusso degli eventi & 
\begin{enumerate}
\item L'utente clicca sulla scritta "Password dimenticata?"
\item L'applicazione mostra il form per il reset della password
\item L'utente inserisce la sua email nel campo "Email" della schermata
\item L'utente clicca sul bottone di "Reset"
\end{enumerate}\\
\hline
Condizione d'uscita & L'applicazione mostra un messaggio di successo\\
\hline
Eccezioni & 
\begin{itemize}
\item L'utente non ha inserito credenziali valide
\item L'utente non ha inserito l'email nel campo relativo
\item Il sistema non è in grado di completare la richiesta a causa di un errore interno
\end{itemize} \\
\hline
\end{tabular}
\end{table}

\subsection{Logout}
\begin{table}[H]
\centering
\begin{tabular}{|p{4cm}|p{10cm}|}
\hline
Scenario & Cambio schermata \\
\hline
Condizione d'ingresso & L'utente è nella schermata di Homepage o nella Schermata del Volontario\\
\hline
Flusso degli eventi & 
\begin{enumerate}
\item L'utente clicca sull'icona relativa alla logout in alto a sinistra
\end{enumerate}\\
\hline
Condizione d'uscita & L'applicazione esegue il logout dell'utente e lo reindirizza alla schermata di Login
\\
\hline
Eccezioni & 
\begin{itemize}
\item Il sistema non è in grado di completare la richiesta a causa di un errore interno
\end{itemize} \\
\hline
\end{tabular}
\end{table}

\subsection{Homepage}
\begin{table}[H]
\centering
\begin{tabular}{|p{4cm}|p{10cm}|}
\hline
Scenario & Cambio schermata \\
\hline
Condizione d'ingresso & L'utente è nella schermata di Homepage\\
\hline
Flusso degli eventi & 
\begin{enumerate}
\item L'utente clicca sull'icona relativa alla schermata scelta in alto a destra (Numeri Utili o Scanner QR)
\end{enumerate}\\
\hline
Condizione d'uscita & L'applicazione reindirizza l'utente nella schermata scelta
\\
\hline
Eccezioni & 
\begin{itemize}
\item Il sistema non è in grado di completare la richiesta a causa di un errore interno
\end{itemize} \\
\hline
\end{tabular}
\end{table}

\begin{table}[H]
\centering
\begin{tabular}{|p{4cm}|p{10cm}|}
\hline
Scenario & Scelta documenti \\
\hline
Condizione d'ingresso & L'utente è nella schermata di Homepage\\
\hline
Flusso degli eventi & 
\begin{enumerate}
\item L'utente clicca sull'icona relativa al menu in alto a destra della schermata
\item L'applicazione mostra le date relative allo storico dell'utente
\item L'utente clicca sulla data scelta
\end{enumerate}\\
\hline
Condizione d'uscita & L'applicazione aggiorna i dati del cittadino mostrando quelli relativi alla data scelta
\\
\hline
Eccezioni & 
\begin{itemize}
\item Il cittadino non dispone di altre date tra cui scegliere oltre a quella attuale
\item Il sistema non è in grado di completare la richiesta a causa di un errore interno
\end{itemize} \\
\hline
\end{tabular}
\end{table}

\begin{table}[H]
\centering
\begin{tabular}{|p{4cm}|p{10cm}|}
\hline
Scenario & Codice QR \\
\hline
Condizione d'ingresso & L'utente è nella schermata di Homepage, nella tab Codice QR\\
\hline
Flusso degli eventi & 
\begin{enumerate}
\item L'utente clicca sul bottone:
\begin{itemize}
\item "Apri"
\item "Salva"
\item "Stampa"
\end{itemize}
\item L'applicazione esegue l'azione richiesta
\end{enumerate}\\
\hline
Condizione d'uscita & L'applicazione:
\begin{itemize}
\item apre il codice QR come immagine
\item salva nel dispositivo il codice QR come immagine
\item apre la schermata di stampa per il codice QR in formato .pdf
\end{itemize}\\
\hline
Eccezioni & 
\begin{itemize}
\item Il sistema non è in grado di completare la richiesta a causa di un errore interno
\end{itemize} \\
\hline
\end{tabular}
\end{table}

\begin{table}[H]
\centering
\begin{tabular}{|p{4cm}|p{10cm}|}
\hline
Scenario & Informazioni \\
\hline
Condizione d'ingresso & L'utente è nella schermata di Homepage, nella tab Informazioni\\
\hline
Flusso degli eventi & 
\begin{enumerate}
\item L'utente sceglie il documento di interesse:
\begin{itemize}
\item Dati
\item Badge
\item CIS
\item Braccialetto
\end{itemize}
\item L'utente clicca sull'icona relativa a:
\begin{itemize}
\item "Apri"
\item "Salva"
\item "Stampa"
\end{itemize}
\item L'applicazione genera il documento ed esegue l'azione richiesta
\end{enumerate}\\
\hline
Condizione d'uscita & L'applicazione:
\begin{itemize}
\item apre il documento in formato .pdf
\item salva nel dispositivo il documento in formato .pdf
\item apre la schermata di stampa per documento in formato .pdf
\end{itemize}\\
\hline
Eccezioni & 
\begin{itemize}
\item Il sistema non è in grado di completare la richiesta a causa di un errore interno
\end{itemize} \\
\hline
\end{tabular}
\end{table}

\begin{table}[H]
\centering
\begin{tabular}{|p{4cm}|p{10cm}|}
\hline
Scenario & Covid19 - QR code\\
\hline
Condizione d'ingresso & L'utente è nella schermata di Homepage, nella tab Covid19\\
\hline
Flusso degli eventi & 
\begin{enumerate}
\item L'utente clicca sul bottone:
\begin{itemize}
\item "Apri"
\item "Salva"
\item "Stampa"
\end{itemize}
\item L'applicazione esegue l'azione richiesta
\end{enumerate}\\
\hline
Condizione d'uscita & L'applicazione:
\begin{itemize}
\item apre il codice QR come immagine
\item salva nel dispositivo il codice QR come immagine
\item apre la schermata di stampa per il codice QR in formato .pdf
\end{itemize}\\
\hline
Eccezioni & 
\begin{itemize}
\item Il sistema non è in grado di completare la richiesta a causa di un errore interno
\end{itemize} \\
\hline
\end{tabular}
\end{table}

\begin{table}[H]
\centering
\begin{tabular}{|p{4cm}|p{10cm}|}
\hline
Scenario & Covid19 - Storico\\
\hline
Condizione d'ingresso & L'utente è nella schermata di Homepage, nella tab Covid19\\
\hline
Flusso degli eventi & 
\begin{enumerate}
\item L'utente sceglie dalla lista l'evento di interesse
\item L'utente clicca sul bottone "Apri"
\item L'applicazione esegue l'azione richiesta
\end{enumerate}\\
\hline
Condizione d'uscita & L'applicazione apre il documento richiesto nella pagina web di provenienza\\
\hline
Eccezioni & 
\begin{itemize}
\item L'utente non ha alcun dato nello storico Covid
\item Il sistema non è in grado di completare la richiesta a causa di un errore interno
\end{itemize} \\
\hline
\end{tabular}
\end{table}

\subsection{Schermata del Volontario}
\begin{table}[H]
\centering
\begin{tabular}{|p{4cm}|p{10cm}|}
\hline
Scenario & Ricerca cittadino \\
\hline
Condizione d'ingresso & Il volontario ha scaricato l'applicazione o ha eseguito l'accesso attraverso il web browser \newline
Il volontario ha già un account \\
\hline
Flusso degli eventi & 
\begin{enumerate}
\item Il volontario inserisce il nome (o il codice fiscale) del cittadino di sua competenza nella barra di ricerca che presenta la scritta "Cerca..."
\item L'applicazione mostra i cittadini che soddisfano la ricerca del volontario
\item Il volontario seleziona il cittadino (o i cittadini) desiderato cliccando il pulsante "Aggiungi"
\end{enumerate}\\
\hline
Condizione d'uscita & L'applicazione aggiunge alla lista il cittadino selezionato e lo mostra nella schermata\\
\hline
Eccezioni & 
\begin{itemize}
\item Il volontario ha inserito un nome (o un codice fiscale) inesistente
\item Il cittadino non fa parte delle competenze del volontario
\item Il sistema non è in grado di completare la richiesta a causa di un errore interno
\end{itemize} \\
\hline
\end{tabular}
\end{table}

\begin{table}[H]
\centering
\begin{tabular}{|p{4cm}|p{10cm}|}
\hline
Scenario & Rimozione cittadino \\
\hline
Condizione d'ingresso & Il volontario ha scaricato l'applicazione o ha eseguito l'accesso attraverso il web browser \newline
Il volontario ha già un account\newline
Il volontario ha selezionato un cittadino attraverso la barra di ricerca\\
\hline
Flusso degli eventi & 
\begin{enumerate}
\item Il volontario seleziona il cittadino di interesse tra quelli presenti nella lista
\item Il volontario clicca sull'icona ad X nella scheda relativa al cittadino
\end{enumerate}\\
\hline
Condizione d'uscita & L'applicazione rimuove dalla lista il cittadino selezionato\\
\hline
Eccezioni & 
\begin{itemize}
\item Il sistema non è in grado di completare la richiesta a causa di un errore interno
\end{itemize} \\
\hline
\end{tabular}
\end{table}

\begin{table}[H]
\centering
\begin{tabular}{|p{4cm}|p{10cm}|}
\hline
Scenario & Scelta documenti cittadino \\
\hline
Condizione d'ingresso & Il volontario ha scaricato l'applicazione o ha eseguito l'accesso attraverso il web browser \newline
Il volontario ha già un account\newline
Il volontario ha selezionato un cittadino attraverso la barra di ricerca \\
\hline
Flusso degli eventi & 
\begin{enumerate}
\item Il volontario seleziona il cittadino di interesse tra quelli presenti nella lista
\item Il volontario clicca sull'icona rappresentante il menu nella scheda relativa al cittadino
\item Il volontario clicca sulla data di interesse
\end{enumerate}\\
\hline
Condizione d'uscita & L'applicazione aggiorna i dati del cittadino mostrando quelli relativi alla data scelta\\
\hline
Eccezioni & 
\begin{itemize}
\item Il cittadino selezionato non dispone di altre date tra cui scegliere oltre a quella attuale
\item Il sistema non è in grado di completare la richiesta a causa di un errore interno
\end{itemize} \\
\hline
\end{tabular}
\end{table}

\begin{table}[H]
\centering
\begin{tabular}{|p{4cm}|p{10cm}|}
\hline
Scenario & Stampa singola \\
\hline
Condizione d'ingresso & Il volontario ha scaricato l'applicazione o ha eseguito l'accesso attraverso il web browser \newline
Il volontario ha già un account\newline
Il volontario ha selezionato un cittadino attraverso la barra di ricerca \\
\hline
Flusso degli eventi & 
\begin{enumerate}
\item Il volontario seleziona il cittadino di interesse tra quelli presenti nella lista
\item Il volontario seleziona il documento di interesse tra Badge, CIS e Braccialetto
\item Il volontario clicca sul pulsante "Stampa"
\end{enumerate}\\
\hline
Condizione d'uscita & L'applicazione mostra la schermata di stampa del documento selezionato in formato .pdf\\
\hline
Eccezioni & 
\begin{itemize}
\item Il sistema non è in grado di completare la richiesta a causa di un errore interno
\end{itemize} \\
\hline
\end{tabular}
\end{table}

\begin{table}[H]
\centering
\begin{tabular}{|p{4cm}|p{10cm}|}
\hline
Scenario & Stampa multipla \\
\hline
Condizione d'ingresso & Il volontario ha scaricato l'applicazione o ha eseguito l'accesso attraverso il web browser \newline
Il volontario ha già un account\newline
Il volontario ha selezionato più cittadini attraverso la barra di ricerca \\
\hline
Flusso degli eventi & 
\begin{enumerate}
\item Il volontario clicca sul pulsante in basso a destra dello schermo che presenta l'icona di stampa
\item Il volontario seleziona il documento di interesse cliccando sull'icon corrispondente
\end{enumerate}\\
\hline
Condizione d'uscita & L'applicazione mostra la schermata di stampa dei documenti selezionati in formato .pdf\\
\hline
Eccezioni & 
\begin{itemize}
\item Il sistema non è in grado di completare la richiesta a causa di un errore interno
\end{itemize} \\
\hline
\end{tabular}
\end{table}

\subsection{Numeri Utili}
\begin{table}[H]
\centering
\begin{tabular}{|p{4cm}|p{10cm}|}
\hline
Scenario & Contatti \\
\hline
Condizione d'ingresso & L'utente ha scaricato l'applicazione o ha eseguito l'accesso attraverso il web browser\newline
L'utente ha già un account creato dal volontario incaricato\newline
L'utente è nella schermata Numeri Utili, nella tab Contatti\\
\hline
Flusso degli eventi & 
\begin{enumerate}
\item L'utente sceglie il contatto desiderato tra MMG e volontario
\item L'utente clicca sulla forma di comunicazione desiderata (numero telefonico o email)
\end{enumerate}\\
\hline
Condizione d'uscita & L'applicazione apre l'applicazione del dispositivo in grado di gestire la richiesta\\
\hline
Eccezioni & 
\begin{itemize}
\item Il sistema non è in grado di completare la richiesta a causa di un errore interno
\end{itemize} \\
\hline
\end{tabular}
\end{table}

\begin{table}[H]
\centering
\begin{tabular}{|p{4cm}|p{10cm}|}
\hline
Scenario & Contatti \\
\hline
Condizione d'ingresso & L'utente ha scaricato l'applicazione o ha eseguito l'accesso attraverso il web browser\newline
L'utente è nella schermata Numeri Utili, nella tab Emergenza\\
\hline
Flusso degli eventi & 
\begin{enumerate}
\item L'utente sceglie il contatto desiderato tra Carabinieri, Polizia di Stato, Vigili del Fuoco, Guardia di Finanza e Emergenza Sanitaria
\item L'utente clicca sul contatto desiderato
\end{enumerate}\\
\hline
Condizione d'uscita & L'applicazione apre l'applicazione del dispositivo in grado di gestire la richiesta\\
\hline
Eccezioni & 
\begin{itemize}
\item Il sistema non è in grado di completare la richiesta a causa di un errore interno
\end{itemize} \\
\hline
\end{tabular}
\end{table}

\subsection{QR Code Scanner}
\begin{table}[H]
\centering
\begin{tabular}{|p{4cm}|p{10cm}|}
\hline
Scenario & Scansione QR \\
\hline
Condizione d'ingresso & L'utente ha scaricato l'applicazione\newline
L'utente è nella schermata QR Code Scanner\\
\hline
Flusso degli eventi & 
\begin{enumerate}
\item L'utente clicca sul pulsante "Scan"
\item L'applicazione attiva la camera del dispositivo in attesa della scansione
\item L'utente esegue la scansione di un codice QR
\end{enumerate}\\
\hline
Condizione d'uscita & L'applicazione mostra i dati relativi al codice QR scansionato\\
\hline
Eccezioni & 
\begin{itemize}
\item L'utente cerca di eseguire la scansione di un codice QR che non soddisfa i requisiti dell'applicazione
\item Il sistema non è in grado di completare la richiesta a causa di un errore interno
\end{itemize} \\
\hline
\end{tabular}
\end{table}

\section{User-Flow Diagram}
Il grafico seguente mostra nel dettaglio tutte le possibili interazioni tra le schermate dell'applicazione attraverso un user flow diagram.


\chapter{Conclusione e Sviluppi Futuri}
Il sistema realizzato riesce a concentrare in un unico flusso la fase di inserimento dei dati sanitari e salvavita del cittadino e il loro utilizzo per garantire servizi di utilità e salvaguardia per gli utenti finali. Sebbene realizzato con dati fittizi e senza la presunzione di andare a sostituire quanto già in uso, si propone come un'idea con cui poter sviluppare, in futuro, un sistema complessivo in grado di gestire interamente questo tipo di informazioni e dati per poterne semplificare l'accesso e l'uso.\newline

Come già analizzato brevemente all'inizio di questa trattazione, lo stato dell'arte presenta numerosi sistemi proprietari che non si interfacciano e rendono spesso complicata l'interazione o l'accesso ai propri cittadini. Realizzato quanto fatto, poniamo le basi per un sistema in grado di interfacciarsi con le differenti realtà presenti.\newline

L'inserimento della schermata relativa al Covid19 risulta sicuramente volto all'utilità che potrebbe conseguirne soprattutto in questo periodo storico in cui avere a portata di mano i documenti relativi a tamponi e vaccinazioni può rappresentare un vantaggio notevole a livello di utilità, ma più che altro mostra come il sistema realizzato sia in grado di integrare, all'occorrenza, nuove funzionalità che possano aiutare i cittadini nella loro vita sanitaria cercando di inglobare in un unico accesso quanti più dati e servizi utili per gli stessi.\newline

Partendo da quanto fatto ci auguriamo che il lavoro realizzato possa fornire spunti interessanti per lo sviluppo futuro di un sistema completo e coerente dotato di tutte le caratteristiche necessarie per aiutare il cittadino nella gestione e nell'accesso a dati fondamentali come quelli sanitari.

\chapter{Bibliografia}
Di seguito le fonti utilizzate per la stesura del documento e la realizzazione del progetto:
\begin{itemize}
\item \href{https://www.medicivolontaritaliani.org/}{Medici Volontari Italiani};
\item \href{https://www.comune.milano.it/aree-tematiche/servizi-sociali/raccolta-dati-personali-per-interventi-di-emergenza}{Comune di Milano};
\end{itemize}

\end{document}